\HeaderA{StartPE}{Start/Stop Parallel Execution}{StartPE}
\aliasA{ParallelR}{StartPE}{ParallelR}
\aliasA{StopPE}{StartPE}{StopPE}
\aliasA{taskPR}{StartPE}{taskPR}
\keyword{programming}{StartPE}
\begin{Description}\relax
StartPE starts the parallel engine.  If spawn is true, then the worker
processes are spawned (using \code{MPI\_COMM\_Spawn} from MPI-2).
StopPE stops the parallel engine.  This call blocks until all jobs
are finished.
\end{Description}
\begin{Usage}
\begin{verbatim}
StartPE(num = 2, port = 32000, verbose=0, spawn=TRUE)
StopPE()
\end{verbatim}
\end{Usage}
\begin{Arguments}
\begin{ldescription}
\item[\code{num}] number of worker processes to use 
\item[\code{port}] the TCP port to use for communicating with workers 
\item[\code{verbose}] the verbose level: 0, 1, or 2 at the moment 
\item[\code{spawn}] should the worker processes be spawned? 
\end{ldescription}
\end{Arguments}
\begin{Details}\relax
The parallel engine must be enabled before instructions can be executed
in parallel.  The engine can be stopped and restarted with a different
number of worker processes, if desired.
The parallel engine consists of \code{num} + 1 threads and \code{num}
worker processes.  The worker processes can either be spawned (done
through an MPI call) or connected manually.  If StartPE is run with
spawn = FALSE, then it will block until \code{num} worker processes
have connected.
\end{Details}
\begin{SeeAlso}\relax
\code{\LinkA{PE}{PE}}  For executing jobs in the background/parallel.
\code{\LinkA{POBJ}{POBJ}}  For returning background/parallel jobs to the main process.
\code{\LinkA{StartWorker}{StartWorker}}  For manually starting worker processes.
\end{SeeAlso}
\begin{Examples}
\begin{ExampleCode}
## Not run: 
# If you have MPI running
StartPE(2)

x = matrix(rnorm(128 * 128), 128, 128)

PE( a <- svd(x) )
PE( b <- solve(x) )
PE( y <- b %*% a$u )
POBJ( y )
str(y)
StopPE()
## End(Not run)
\end{ExampleCode}
\end{Examples}

