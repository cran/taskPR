\HeaderA{StartWorker}{Start Parallel-R Worker Process}{StartWorker}
\keyword{programming}{StartWorker}
\begin{Description}\relax
Attempts to connect to the given host and establish itself as a worker
process.  This function is called automatically when worker processes
are spawned by the main process.
\end{Description}
\begin{Usage}
\begin{verbatim}
StartWorker(host = "localhost", port=32000, retries=2, sleeptime=1, quiet=TRUE)
\end{verbatim}
\end{Usage}
\begin{Arguments}
\begin{ldescription}
\item[\code{host}] name of the machine that the main/controller process is on 
\item[\code{port}] the (TCP/IP) port number to connect to 
\item[\code{retries}] the number of times to retry making the connection 
\item[\code{sleeptime}] how long (in seconds) to sleep between connection tries 
\item[\code{quiet}] should the worker process supress most logging messages? 
\end{ldescription}
\end{Arguments}
\begin{Details}\relax
The only time a user should call this function is when they started the
parallel engine on the main process using the spawn=FALSE option to
StartPE.  In that case, the main process will block waiting for the
worker processes to connect.  The user must run the appropriate number
of worker processes and have them call this function.
\end{Details}
\begin{SeeAlso}\relax
\code{\LinkA{StartPE}{StartPE}}  For enabling the parallel engine.
\code{\LinkA{PE}{PE}}        For running parallel jobs.
\code{\LinkA{POBJ}{POBJ}}  For returning background jobs to the main process.
\end{SeeAlso}
\begin{Examples}
\begin{ExampleCode}
## Not run: 
# If you have MPI running
StartPE(2)

x = matrix(rnorm(128 * 128), 128, 128)

PE( a <- svd(x) )
PE( b <- solve(x) )
PE( y <- b %*% a$u )
POBJ( y )
str(y)
StopPE()
## End(Not run)
\end{ExampleCode}
\end{Examples}

